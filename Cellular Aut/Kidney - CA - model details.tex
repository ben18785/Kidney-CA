\documentclass[pdftex,10pt,a4paper]{article}
\title{\textbf{Kidney Morphogenesis Cellular Automaton - Model Details}\newline }
\author{}
\usepackage[authoryear]{natbib}
\usepackage[titletoc,toc]{appendix}
\usepackage[pdftex]{graphicx}
\usepackage{url,times}
\usepackage{graphicx}
\usepackage{epstopdf}
\usepackage{amsmath}
\usepackage[all]{xy}
\usepackage{pxfonts}
\usepackage{colortbl}
\usepackage{color}
\usepackage{subfigure}
\usepackage{gensymb}
\usepackage{ctable}
\usepackage[justification=centering]{caption}[2007/12/23]
\usepackage{longtable}
\usepackage{pst-func}
\usepackage{pst-math}
\setlength{\parindent}{0.0in}
\setlength{\parskip}{0.1in}
\usepackage[margin=0.5in]{geometry}
\renewcommand{\bibname}{Works Cited}
\usepackage{listings}
\usepackage{setspace}
\usepackage{algorithm}
\usepackage{bbm}


\newcommand{\HRule}{\rule{\linewidth}{0.5mm}}
\begin{document}

\maketitle
\doublespacing
\section{Introduction}
The aim of this model is to recapitulate the development of the Uretic Bud \textit{in vivo} in 2D. It is also hoped that the model may be useful for the study of \textit{in vitro} explants. The details of the model are described in brief below:
\begin{itemize}
\item A 2 dimensional rectangular domain, where the points in the array represent the presence of:
\begin{itemize}
\item Epithelium cells - which consume GDNF
\item Metanephric mesenchymal cells (MM) - which produce GDNF
\item Extracellular matrix (ECM) - which allow for free diffusion of GDNF
\end{itemize}
\item Epithelium cells comprising either:
\begin{itemize}
\item A \textit{flat} membrane at the base of the simulation domain (aimed at recapitulating the \textit{in vivo} initial conditions)
\item A mass of epithelium cells with in general, curved boundaries, suspended towards the centre of the simulation domain (representative of the \textit{in vitro} explant experiment initial conditions)
\end{itemize}
\item A diffuse collection of mesenchymal cells initially separate from the epithelium
\end{itemize}

\section{GDNF field}
Cells in the MM produce GDNF, and it diffuses freely across the ECM layer, and through the epithelium (being consumed by the latter). The reaction-diffusion type equation here will be assumed to be in equilibrium, since the diffusive timescale of GDNF is much less than that of the timescale of cell division. As such, the following is the form of the equation being solved:

\begin{equation}\label{eq:diffnorm}
D_G \nabla^2 = \Phi_G
\end{equation}

Where in (\ref{eq:diffnorm}) $D_G$ refers to the diffusion coefficient for GDNF, and $\Phi_G$ is the local rate of GDNF production or consumption (dependent on whether the cell in question is epithelium or mesenchyme). Specifically the rate of GDNF consumption is assumed to have the following form:

\begin{equation} \label{eq:production}
\Phi_G =\begin{cases}
K_G G, & \text{for epithelium}.\\
-\rho_G, & \text{for mesenchyme}.\\
0, & \text{for the extracellular matrix}.
\end{cases}
\end{equation}

It is assumed that the rate of GDNF consumption is linearly-dependent on the concentration of substrate. This assumption is likely violated when the GDNF concentration is high, and the cell Ret-receptors are saturated. In  later versions of this model, it may be better to assume a Hill-type reaction rate. (Although it is unclear as to what range of concentrations of GDNF are likely to saturate the Ret receptors, and whether these are likely to be encountered either \textit{in vivo} or \textit{in vitro}.)

The equation in (\ref{eq:diffnorm}) is non-dimensionalised using the following transformations:
\begin{equation}\label{eq:difftrans1}
\eta = \frac{x}{\Delta}
\end{equation}

\begin{equation}\label{eq:difftrans2}
g = \frac{G}{G_x}
\end{equation}

Where in (\ref{eq:difftrans1}) $\Delta$ refers to the typical cell dimensions (approximated as 5 $\mu m$),and $G_x$ is the concentration of GDNF typically found \textit{in vivo}. These transformations result in the following non-dimensional form of the steady-state reaction-diffusion equation:

\begin{equation}\label{eq:diff-dimensionless}
\nabla_\eta^2 g = \frac{\phi_g}{d_g}
\end{equation}

Where in (\ref{eq:diff-dimensionless}), $d_g = \frac{D_G}{K_G \Delta^2}$, and $\phi_g = \frac{\Phi_G}{K_G G_x}$.

The boundary conditions which are assumed are: no-flux at the base of the Wolffian Duct, and periodic boundary conditions at either width. A finite difference approximation is used, with the no-flux boundary conditions at the top and bottom of the domain represented by the following relations for the non-dimensional GDNF field:

\begin{align*}
g_{0,j}& = g_{1,j}\\
g_{M,j}& = g_{M+1,j}\\
\end{align*}

where this holds $\forall j = 1,...,N$. Similarly, for the periodic boundary conditions:

\begin{align*}
g_{i,0}& = g_{i,N}\\
g_{i,N+1}& = g_{i,1}\\
\end{align*}

where this holds $\forall i = 1,...,M$.

The finite-difference approximation for solving the steady-state reaction-diffusion equation is hence of the form:

\begin{equation}\label{eq:finitediff}
g_{i+1,j} + g_{i-1,j} + g_{i,j+1} + g_{i,j-1} - (4 + \epsilon_{i,j})g_{i,j} = -\psi_{i,j}
\end{equation}

Where in (\ref{eq:finitediff}):
\begin{equation}
\epsilon_{i,j} =\begin{cases}
\frac{1}{d_g}, & \text{for epithelium}.\\
0, & \text{for mesenchyme}.\\
0, & \text{for the extracellular matrix}.
\end{cases}
\end{equation}

Similarly, in (\ref{eq:finitediff}):
\begin{equation} \label{eq:psi}
\psi_{i,j} =\begin{cases}
0, & \text{for epithelium}.\\
\frac{\gamma}{d_g}, & \text{for mesenchyme}.\\
0, & \text{for the extracellular matrix}.
\end{cases}
\end{equation}

Where in (\ref{eq:psi}), $\gamma = \frac{\rho_G}{K_G G_x}$ captures the relative rate of GDNF production by the mesenchyme compared to the rate of GDNF consumption by epithelium under normal cellular conditions. It is assumed that $\gamma = 1$ in the simulations, in the absence of information regarding the relative importance of these two mechanisms. The diffusion rate constant, $D_G$, is assumed to be 10 $\mu m^2 s^{-1}$, and $K_G = 4 \times 10^{-2} s^{-1}$ (for this I currently have not thought of a way to estimate this parameter value - the \textit{in vitro} experiments I have read have, thus far, not provided sufficient information to estimate this parameter); making $d_g = 100$. 

In the simulation, it is assumed that the GDNF field does not vary significantly within a particular time step. As such (in order to improve computational speed), the GDNF field is only updated at the end of each time step, not after each cell movement or proliferation. Ultimately however, it will be important to test whether the model conclusions are sensitive to this assumption.

\section{Epithelium cell behaviour}
\subsection{Algorithm governing cellular behaviour}
In this section I describe the behaviours of the epithelium cells in the cellular automaton model. In the simulation, each epithelium cell is visited in a random order, (actually epithelium and mesenchyme are updated together, so the randomised order corresponds to both epithelium and mesenchymal cells), and updated at each discrete time step according to the following pseudo-algorithm:

\begin{itemize}
\item \textit{Available cells} - Are neighbouring cells available for movement or proliferation? If yes, proceed to step 2. If no, move on to next cell. Whether a cell is available depends on the specific rules in place (see section \ref{sec:rule_available} for more details).
\item \textit{Move or proliferate} - Draw a random number $X\sim unif(0,1)$. If $P_{move}>X$ then proceed to \textit{move}. If not, proceed to \textit{proliferate}.
\item \textit{Move} 
\begin{enumerate}
\item \textit{Calculate the probability that a move takes place, $P_m$} - Based on the specific rules in place calculate the probability that a move takes place (see section \ref{sec:rule_pmove} for more details).
\item \textit{Does a move take place?} - Draw a random number $X\sim unif(0,1)$. If $P_{m}>X$ then proceed to next step. Otherwise, consider next epithelium cell.
\item \textit{Calculate the weights for probabilities of moving to each of the allowed cells} - Based on the specific rules being used, calculate a set of weights which will be used (in the next step) to calculate the probability of moving to each of the allowed cells (see section \ref{sec:rule_selection} for more details about the options for rules used here). 
\item \textit{Calculate probability of moving to each of the available cells} - using the weights from the last step as parameters in a \textit{Dirichlet} distribution, calculate a probability of moving to each of the cells. 
\item \textit{Choose amongst the available cells in accordance to their probability and move the cell in question}
\end{enumerate} 
\item \textit{Proliferate}
\begin{enumerate}
\item \textit{Calculate the probability that a proliferation takes place, $P_p$} - Based on the specific rules in place calculate the probability that a proliferation takes place (see section \ref{sec:rule_pmove} for more details).
\item \textit{Does a proliferation take place?} - Draw a random number $X\sim unif(0,1)$. If $P_{p}>X$ then proceed to next step. Otherwise, consider next epithelium cell.
\item \textit{Calculate the weights for probabilities of creating a daughter cell in each of the allowed cells} - Based on the specific rules being used (see section \ref{sec:rule_selection} for more details), calculate a set of weights which will be used (in the next step) to calculate the probability of creating a daughter cell in each of the allowed cells (see section \ref{sec:rule_pmove} for more details).
\item \textit{Calculate probability of proliferating into each of the available cells} - using the weights from the last step as parameters in a \textit{Dirichlet} distribution, calculate a probability of proliferating into each of the cells. 
\item \textit{Choose amongst the available cells in accordance to their probability and create a daughter cell in the selected location}
\end{enumerate} 
\item Consider next cell, returning to first step.
\end{itemize}

The program has been created in Matlab which allows for as many permutations of the rules to be tried by the individual carrying out the simulation as possible. However, there are some combinations which are not allowed due to internal inconsistency amongst them. The idea is to allow the model's results to be easily tested for robustness to assumptions made.

\subsection{Rules governing whether neighbouring cells are available}\label{sec:rule_available}
In this section I will detail the various different rules which govern which cells are available to move or proliferate into. These rules are allowed to be different for moving and proliferating.

\begin{enumerate}
\item \textit{Number of neigbours} - the choice here governs whether 4 (up, down, left, and right) or 8 (all points of a compass) are surveyed as potential candidates for a move or proliferation.
\item \textit{Connectivity of cells}
\begin{enumerate}
\item All vacant cells are allowed.
\item Moves or proliferations are allowed into vacant cells only if the active cell (the one being moved, or the daughter cell being created) are connected to other cells. Here connectivity means that there is at least one 4-neighbour (up, down, left or right) which is occupied by an epithelium cell \label{item:sticky1}
\item Same as above, but more stringent. Only allow moves if all cells are connected. This allows for the possiblity of a move leaving one cell unconnected.\label{item:sticky2}
\item All vacant cells and mesenchyme are allowed as possible move/proliferation locations. If a movement into a cell occupied by a mesenchymal cell occurs, then the mesenchyme cell is removed from the simulation.\label{item:mes1}
\item Same as above, although only allow movement into a space if that active cell remains connected. \label{item:sticky3}
\item Same as above, although instead of killing mesenchyme cell (upon movement into its space), move the mesenchyme into a vacant space neighbouring it. If there are no vacant space available for the mesenchyme cell to move into, then the move is not allowed. If multiple spaces are available for the cell to move into, then choose one of them at random.\label{item:sticky4}
\end{enumerate}
\end{enumerate}

The above rules \ref{item:sticky1}, \ref{item:sticky2}, \ref{item:sticky3} and \ref{item:sticky4} are phenomenlogical, but are aimed at mimicking the cell-cell adhesion experienced by the epithelium. Rules \ref{item:mes1}, \ref{item:sticky3}, \ref{item:sticky4} are included as to allow an interaction between the epithelium cells and the mesenchyme, when the former move into the area occupied by the latter.

\subsection{Rules governing the calculation of $P_{m}$ or $P_p$ - the probability of a move or proliferation occurring after the action has been chosen}\label{sec:rule_pmove}
In this section I describe the available choices of model rules which govern the calculation of the probability of a move/proliferation occurring takes place. To be clear, this is the step after the choice has been made to either go down the algorithmic branches corresponding to \textit{Move} or \textit{Proliferate} respectively. The rules available for calculating whether or not a move takes place are given below:

\begin{enumerate}
\item The probability of a move/proliferation is a constant specified by the user.
\item The probability of a move/proliferation is a positive function of the local GDNF concentration (the level of GDNF for that cell at that grid point). The equation used here to specify the probability is that of a probit model 
\begin{equation}\label{eq:probit}
P(action) = \Phi(c_1 + c_2 g)
\end{equation}
Where in (\ref{eq:probit}), $\Phi$ is the standard normal CDF, and $action$ can either be a move or a proliferation event.\label{item:pmove_GDNF}
\item Move/proliferation probability is determined by the sum of all local positive GDNF gradients, between each of the available neighbours and the current location. Again, the model used here for the probability is the probit model:
\begin{equation}\label{eq:probit2}
P(action) = \Phi(c_1 + c_2 \sum\limits_{allowed}} (g^{neighbours} - g^{current})\times \mathbbm{1}(g^{neighbours} > g)
\end{equation}
Where in (\ref{eq:probit2}), the indicator function $\mathbbm{1}(g^{neighbours} > g)$, is equal to one if the gradient in GDNF allowed by the move is positive, and zero otherwise.\label{item:pmove_gradient}
\end{enumerate}

Rule \ref{item:pmove_GDNF} is aimed at allowing more cellular activity (moves or proliferations) in areas where GDNF concentration is higher. Rule \ref{item:pmove_gradient} allows more cellular activity in areas where there is more GDNF to be gained by a potential move or proliferation. I am less sure as to the biological realism of this latter rule.

\subsection{Rules governing selection of move target or daughter cell location}\label{sec:rule_selection}
This last section on rules is that governing how to determine the location (amongst the available, allowed neighbouring cells) into which either a cell should move or create a daughter cell. The rules governing how the  probabilities of choosing a given target location are given below:

\begin{enumerate}
\item The probabilities of choosing each of the possible target cell are equal, and given by:

\begin{equation}\label{eq:tot_target}
P(target) = \frac{1}{N_{target}} 
\end{equation}
Where in (\ref{eq:tot_target}), $N_{target}$ specifies the total number of allowed possible target cells.

\item The weights given to a specific target cell are given by:
\begin{equation}
W(target) = \Phi (c_3 + c_4(g^{target} - g^{current}))
\end{equation}

The weights across all targets are then used as weights in a Dirichlet distribution to calculate probabilities which sum to 1; with the higher weights corresponding to higher probabilities.

\item The weights given to a specific target cell are given by:
\begin{equation}
W(target) = \Phi (c_3 + c_4\frac{g^{target} - g^{current}}{g^{current}})
\end{equation}

The weights across all targets are then used as weights in a Dirichlet distribution to calculate probabilities which sum to 1; with the higher weights corresponding to higher probabilities.

\item The probabilities of choosing a specific target cell are given by a multinomial logistic distribution:

\begin{equation} \label{eq:multilogit}
P(target_n) = \frac{exp(c_3 + c_4(g^{target} - g^{current}))}{\sum\limits_{all targets} exp(c_3 + c_4(g^{target} - g^{current}))}
\end{equation}
Where in (\ref{eq:multilogit}), the benefit of this particular form is that the probabilities naturally sum to 1.

\end{enumerate}


\end{document}_